% ----------
% Vidith Balasa
% Resume (2024)
% ----------

\documentclass{resume} % Use the custom resume.cls style

\usepackage[left=0.4 in,top=0.4in,right=0.4 in,bottom=0.4in]{geometry} % Document margins
\newcommand{\tab}[1]{\hspace{.2667\textwidth}\rlap{#1}} 
\newcommand{\itab}[1]{\hspace{0em}\rlap{#1}}
\name{Vidith Balasa}

\address{
    (559) 317-5372 \\
    \href{mailto:vidithbalasa@gmail.com}{vidithbalasa@gmail.com} \\ 
    \href{https://linkedin.com/in/vidithbalasa}{linkedin.com/in/vidithbalasa} \\
    \href{https://github.com/vidithbalasa}{github.com/vidithbalasa}
}
\begin{document}

%----------------------------------------------------------------------------------------
%	EDUCATION
%----------------------------------------------------------------------------------------
\begin{rSection}{Education}
    {\bf Master of Science in Data Science}, University of San Francisco \hfill {Jul 2023 - Jun 2024}
    \\Relevant Coursework: Machine Learning, Deep Learning,
    NLP, Time Series, Distributed Data Systems

    \noindent{\bf Bachelor of Arts in Philosophy}, University of California, Santa Cruz \hfill {Sep 2018 - Jun 2022}
\end{rSection}

%----------------------------------------------------------------------------------------
%	WORK EXPERIENCE
%----------------------------------------------------------------------------------------
\begin{rSection}{EXPERIENCE}

    \textbf{Square} \hfill Oct 2023 - Present\\
    Machine Learning Engineer -- Intern \hfill \textit{Remote}
     \begin{itemize}
        \itemsep -3pt {} 
        \item Enhanced revenue-to-ad-cost ratio by over 5\% through the development and training of a sophisticated gradient boosting model on a dataset exceeding 100GB, aimed at predicting customer lifetime value with heightened accuracy
        \item Enhanced performance of the existing lifetime revenue prediction pipeline by over 20\% through strategic integration of Prefect 2 frameworks, streamlining deployment of ml models into production
        \item Obtained shareholder approval for production deployment by concisely presenting the model's monetary benefits
     \end{itemize}
     
    \textbf{Siemens} \hfill May 2023 - Jul 2023\\
    Software Engineer -- Intern \hfill \textit{Remote}
     \begin{itemize}
        \itemsep -3pt {} 
        \item Deployed a cutting-edge LLM troubleshoot solution, achieving a 15\% reduction in customer complaints by enhancing the accuracy and response speed of our support systems
        \item Conducted comprehensive research on emerging deep learning and generative AI technologies, contributing to the ongoing integration of cutting edge technologies into the company's product offerings
        \item Effectively communicated complex project milestones and results to a broad range of high level executives
     \end{itemize}
    
    \textbf{Divercity} \hfill Jun 2022 - Sep 2022\\
    Software Engineer -- Intern \hfill \textit{Remote}
     \begin{itemize}
        \itemsep -3pt {} 
        \item Ensured operational integrity of image recognition models deployed on AWS, maintaining uptime at a rate of 97\% month-over-month
        \item Spearheaded the interview process and mentoring of two interns from a pool of over five candidates, playing a crucial role in their development and successful integration into key projects
     \end{itemize}
    
    \end{rSection}
    

%----------------------------------------------------------------------------------------
% PERSONAL ENDEAVORS
%----------------------------------------------------------------------------------------

\begin{rSection}{PERSONAL ENDEAVORS}
\vspace{-1.25em}
\item \textbf{ScaleAI Hackathon Winner} {One of three top prize winners at the annual Scale AI hackathon. Built an LLM based molecular generation model that increases speed of drug discovery for chemical engineers. Fine tuned embedding models on SMILES notation to generate molecular substructure knowledge from plain text.}
\item \textbf{4 Bit CUDA Kernel} {Built a set of CUDA kernels that allows for quantized 4 bit tensor operations on an Nvidia GPU. Nvidia doesn't currently have any native kernels that allow for 4 bit computation, so I decided to build my own. \href{https://github.com/vidithbalasa/4-bit-Quantized-CUDA-Kernel}{(GitHub Repository)}}
\item \textbf{Fog City Rocketry} {Officer for fog city rocketry club. Built a 3D printed rocket with a custom H-class solid fuel engine that reached 10,000 feet. Worked on IMU and flight computer algorithms.}
\end{rSection} 

%----------------------------------------------------------------------------------------
% SKILLS
%----------------------------------------------------------------------------------------
\begin{rSection}{SKILLS}

\begin{tabular}{ @{} >{\bfseries}l @{\hspace{6ex}} l }
Languages & Python, C, C++, SQL, Julia, Javascript/Typescript, HTML, CSS\\
Frameworks & PyTorch, CUDA, TensorFlow, Scikit-learn, Pandas, Numpy, Flask, React\\
Database Tools & Spark, MySQL, PostgreSQL, MongoDB, Redis, Firebase, BigQuery, Snowflake\\
Misc & git, vim, nvcc, Docker, Amazon Web Services (AWS), Airflow, Prefect\\
\end{tabular}\\
\end{rSection}


\end{document}
